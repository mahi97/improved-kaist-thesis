% -*- TeX:UTF-8 -*-
%%
%% KAIST Thesis/Dissertation Template for LaTeX (ver 0.4) Example
%%
%% @version 0.4.1
%% @author  채승병 Chae,Seungbyung (mailto:chess@kaist.ac.kr)
%% @date    2004. 11. 12.
%% @modifier 신호철 (mailto:h.c.shin@kaist.ac.kr)
%% @moddate 2018. 11. 29
%%
%% @requirement
%% LaTeX2e distributions such as teTeX, fpTeX, teTeX, etc.
%% + HLaTeX version 0.991 or later by 은광희 or HPACK 1.0 by 홍석호
%% : For detailed installation instructions, please refer to http://www.ktug.or.kr.
%%
%% @note
%% This template has been completely rewritten rather than following the format
%% of the widely used thesis template class file by 차재춘.
%% There are many differences in the thesis information input section compared to the old format;
%% please modify it according to the example below.
%%
%%
%% @acknowledgement
%% This sample thesis was provided through the courtesy of Kim Yong-hyun,
%% a doctoral student in the Department of Physics.
%%
%% -------------------------------------------------------------------
%% @information
%% This example file uses hangul-ucs. It is written in UTF-8 input encoding.
%% The hfont of hlatex is not used. --2006/02/11
%% This template was bug-fixed by Professor Kim Min-hyuk of the School of Computing. -- 2016/11/25
%% The format was modified by 신호철 of the Department of Electronics (mainly for the draft version’s thesis display style).

% @class kaist.cls
% @options [default: doctor, korean, final]
% - doctor: Ph.D. program | master: Master's program
% - korean: Korean thesis | english: English thesis
% - final: Final version   | draft: Trial version
% - pdfdoc: If not selected, bookmarks and color links will not be created.

% Indicator for the final version
% N.B. Use \finaltrue for dissertation publication, and use \finalfalse for the proposal and defense
\newif\iffinal
\finaltrue % to set final marker as true -> final
% \finalfalse % set final marker as false -> draft

% If you want to make a PDF document (include bookmark, colorlink)
%\documentclass[doctor,english,final,pdfdoc]{kaist-ucs}
\iffinal
	\documentclass[doctor,english,final,pdfdoc]{kaist-ucs-improved}
	\usepackage{lmodern} % This suppresses font size warnings in final version
\else
	\documentclass[doctor,english,draft,pdfdoc]{kaist-ucs-improved}
\fi

% Enable numbering up to subsubsection
% Increase the number to enable numbering for lower levels: e.g. \paragraph and \subparagraph
\setcounter{secnumdepth}{3}

% Line breaking for long URL's (line is broken also by '-')
\def\UrlBreaks{\do\/\do-\do\_}


% In kaist.cls, the dhucs, ifpdf, and graphicx packages are loaded by default.
% If additional packages are required, uncomment and add them,
%\usepackage{...}

% @command title Thesis title (title of thesis)
% @options [default: (none)]
% - korean: Korean title | english: English title
\title[korean] {통신 효율적인 연합학습}
\title[english]{Communication-efficient Federated Learning}


% @note To force a line break in the title displayed on the cover, insert \linebreak.
%       Do not use \\ or \newline. (Example below)
%
%\title[korean]{탄소 나노튜브의 물리적 특성에 대한\linebreak 이론 연구}
%\title[english]{Theoretical study on physical properties of\linebreak
%                carbon nanotubes}
%
% If you want to begin a new line in the cover, use \linebreak.
% See examples above.
%

% @command author Author's name
% @param   family_name, given_name Enter the family name and given name separately
% @options [default: (none)]
% - korean: Korean name | chinese: Hanja name | english: English name
% If there is no Hanja name, you can leave it blank.
%
%
% If you are a foreigner, write your name in Korean or use your Korean name.
% If you cannot write native characters, you can leave the Chinese field empty.
% Write as follows:
% \author[korean]{family name in korean}{given name in korean}
% \author[chinese]{family name in your native language}{given name in your native language}
% \author[english]{family name in english}{given name in english}
%
\author[korean]{홍}{길 동}
\author[korean2]{홍}{길동}    % Please write the name together (without spaces).
\author[chinese]{}{} % Don't want hanja
\author[english]{Hong}{Gildong}

% @command advisor Advisor's name (multiple allowed)
% @usage   \advisor[options]{...Korean name...}{...English name...}{signed|nosign}
% @options [default: major]
% - major: Primary advisor  | coopr: Co-advisor
\advisor[major]{갑}{Gab}{signed}
\advisor[major2]{갑}{Gab}{signed} % Please write the Korean surname and given name together (without spaces).

% For final draft
\advisorinfo{Professor of Electrical Engineering} % Professor information for the thesis submission approval form, advisor's information   
%\advisor[coopr]{홍 길 동}{Gil-Dong Hong}{nosign}
%\advisor[coopr2]{홍길동}{Gil-Dong Hong}{nosign}    % Please write the Korean surname and given name together (without spaces).
%
% It is not necessary to enter the advisor's Korean name.
% For example, you may use: \advisor[major]{}{Chang, Kee Joo}{signed}
%

% @command department {Department name}{Degree type} - Enter the code according to the following rules
% @command department {department code}{degree field}
%
% department code
% Refer to pages 4-5 of "Instructions for Writing and Submitting Master's/Doctoral Theses"
% or refer to the % @command department in kaist-ucs.cls

% science: Science | engineering: Engineering | business: Business Administration
% For Ph.D. dissertations, it is not necessary to input the degree field.
% If you write a Ph.D. dissertation, you cannot input the degree field.
% The third parameter: a | b | c
% a: Option to only include the department name (if you are only affiliated with a department, you must choose a)
% b: Option to include both the department and the program (if you are affiliated with a program or interdisciplinary major under a department)
% c: Option to include only the program or interdisciplinary major name (without the department name), if applicable
%
% a: it represents only the name of the department. (if you aren't in a program under the department, you must choose a)
% b: it represents the names of the department and the program that is under the department (consider this when you are in a program in addition to the department)
% c: it represents only the name of the program that is under the department (consider this when you are in a program rather than just the department)
\department{EE}{engineering}{a}

% @command studentid Student ID
\studentid{20155185}

% @command referee Examiner (3 for master's, 5 for doctoral)
\referee[1]{갑}
\referee[2]{을}
\referee[3]{병}
\referee[4]{정}
\referee[5]{무}
% \referee[5] {Barack Obama}
% Of course, an English name is available

% @command approvaldate Advisor's thesis approval date
% @param   year,month,day Enter in the order of year, month, day
\approvaldate{2020}{11}{30} % SHOULD BE MODIFIED
\fontsize{17.28pt}{17.28pt}\selectfont % Font size changed from 18pt to 17.28pt to remove warnings

% @command refereedate Examiners' thesis review date
% @param   year,month,day Enter in the order of year, month, day
\refereedate{2020}{11}{30} % SHOULD BE MODIFIED

% @command gradyear Graduation year
\gradyear{2020} % SHOULD BE MODIFIED

\iffinal
	\includeonly{
  sections/abstract,
  sections/introduction,
  sections/chapter1/numbered-chapter,
  sections/chapter2/another-numbered-chapter,
  sections/conclusion,sections/cv
  }
\else
  \includeonly{sections/abstract,
  sections/introduction,
  sections/chapter1/numbered-chapter,
  sections/chapter2/another-numbered-chapter,
  sections/conclusion
  }
\fi

% Start of the main text
\begin{document}
    % The front cover, inner cover, thesis submission approval form,
    % and thesis review approval page are automatically generated
    % if you set the class option to final,
    % and are not generated if you set the option to draft.
    % To modify your major and advisor's information on the thesis submission approval form,
    % please open the class file titled "kaist-ucs" provided with the template
    % and change the section marked with ####################.
    
    % Thesis bibliographic information, abstract, keywords,
    % English abstract, and English keywords (Information of thesis, abstract in Korean,
    % keywords in Korean, abstract in English, keywords in English)
    %% The Korean abstract should not exceed 500 characters and the English abstract should not exceed 300 words.
    %% Include no more than 5 keywords.
    %% Do not use English letters in the Korean abstract.
		\thesisinfo
   
    %\begin{summary}      
    %Over the past decade, carbon nanotubes have emerged as one of the ideal basic materials
    %in the emerging nanotechnology field due to their unique electrical and mechanical properties.
    %Depending on the detailed method of wrapping graphite, their electrical properties vary
    %widely from metallic to semiconducting with a band gap of 1 eV.
    %This thesis examines various physical properties of carbon nanotubes,
    %primarily using first-principles density functional theory and the tight-binding approximation model,
    %addressing electrical properties and their control methods, magnetic properties, and transport characteristics.
    %\end{summary}
   
    %\begin{Korkeyword}
    % A, B, C
    %\end{Korkeyword}

		% Locate abstract in a separate file
		\include{sections/abstract}

    \addtocounter{pagemarker}{1}                 % Consideration for blank page allocation
    \newpage  
  
		\iffinal
			% Generate Table of Contents
			\tableofcontents

			% Generate List of Tables
			\listoftables

			% Generate List of Figures
			\listoffigures
		\else
			\label{paperlastromanpagelabel} % For removing warning in draft mode
			\pagenumbering{arabic} % Draft mode has no page numbering so add it.
		\fi

    % The above three types of contents can also be generated at once using the following command:
    %\makecontents
%% In a thesis written in Korean, do not use English letters in the main text.
%% However, if absolutely necessary, write them in the format 'Korean word (English word)'.
%% The remainder of the text should be written according to the standard LaTeX report class format.
%% However, since the 'part' command has been removed, 'chapter' becomes the highest-level division in the document.
%% You cannot use 'part'

% Chapters in separate files
\include{sections/introduction}
\include{sections/chapter1/numbered-chapter}
\include{sections/chapter2/another-numbered-chapter}
\include{sections/conclusion}


%%
%% Bibliography begins
%% bibliography
%% The above is provided as an example and can be adjusted according to the department or thesis requirements.
%% However, each reference must contain sufficient information.

%Bibliography
{\footnotesize
	\bibliographystyle{splncs03}
	\bibliography{references/refGeneral,references/reference2} % Separate multiple *.bib files with commas
}

%%
%% Acknowledgements begin
%% Acknowledgement
%% Writing acknowledgements is optional.
% @command acknowledgement Acknowledgement
% @options [1 | 2 | 3 |4 ]
% - 1: When both the main text and the acknowledgement are in Korean | 2: When the main text is in Korean but the acknowledgement is in English
% - 3: When both the main text and the acknowledgement are in English | 4: When the main text is in English but the acknowledgement is in Korean

\iffinal
	\acknowledgment[4]
	I express my gratitude.
\fi

%%
%% Curriculum Vitae begins
%% Curriculum Vitae
%% Writing a curriculum vitae is optional. You may modify the CV content as appropriate.
% @command curriculumvitae Curriculum Vitae
% @options [1 | 2 | 3 |4 ]
% - 1: When both the main text and the CV are in Korean | 2: When the main text is in Korean but the CV is in English
% - 3: When both the main text and the CV are in English | 4: When the main text is in English but the CV is in Korean

\curriculumvitae[3]
		% @environment personaldata 개인정보
		% @command     name         이름
		%              dateofbirth  생년월일
		%              birthplace   출생지
		%              domicile     본적지
		%              address      주소지
		%              email        E-mail 주소
		% - 위 6개의 기본 필드 중에 이력서에 적고 싶은 정보를 입력
		% input data only you want
		\begin{personaldata}
				\name       {안 진 현}
				\dateofbirth{1997}{12}{14}
				\birthplace {...}
				\address    { ...}
		 \end{personaldata}

		% @environment education 학력
		% @options [default: (none)] - 수학기간을 입력
		\begin{education}
				\item[2007. 3.\ --\ 2009. 2.] 고등학교 (2년 수료)
				\item[2009. 2.\ --\ 2013. 8.] 한국과학기술원 수리과학과 (학사)
				\item[2013. 9.\ --\ 2016. 2.] 한국과학기술원 수리과학과 (석사)
		\end{education}

		% @environment career 경력
		% @options [default: (none)] - 해당기간을 입력
		\begin{career}
				\item[2013. 9.\ --\ 2016. 2.] 한국과학기술원 수리과학과 일반조교
		\end{career}

		% @environment activity 학회활동
		% @options [default: (none)] - 활동내용을 입력
%%    \begin{activity}
%%        \item J. Choi, \textbf{Yong-Hyun Kim}, K.J. Chang, and D. Tomanek,
%%             \textit{Occurrence of itinerant ferromagnetism in C/BN superlattice
%%             nanotubes}, 5th Asian Workshop on First-Principles Electronic
%%             Structure Calculations, Seoul (Korea), October., 2002.
%%    \end{activity}
%% 학회활동을 쓰고싶으시면, 이 문서와 클래스 문서의 학회활동 부분을 사용하십시오.

		% @environment publication 연구업적
		% @options [default: (none)] - 출판내용을 입력
		\begin{publication}
				\item J. Ahn, \textit{Analysis of Tail Probability of Interference at a Node in 2-dimensional Homogeneous Poisson Point Process}, Master Thesis, Korea Adv. Inst. Science, Techn., Daejeon, Republic of Korea, 2016.
		\end{publication}



\label{paperlastpagelabel}     % <-- Additional part: specify the location of the last page	
%% End of main text
\end{document}

